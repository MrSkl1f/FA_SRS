\section*{ВВЕДЕНИЕ}

В настоящее время социальные сети отвечают за треть всей интернет-рекламы.

Целевая реклама и индивидуализация стоят во главе стратегий продвижения разнообразных товаров. Эта сфера стремительно развивается, и крайне важно быть в курсе последних изменений.

Исследования в этой области привлекли значительное внимание по двум основным причинам. 

Во-первых, обилие информации о продукте, доступной для клиентов, непрерывно увеличивается. Соответственно, важно помочь им разобраться в этой информации, чтобы найти наиболее подходящий для них товар или услугу.

Во-вторых, понимание потребностей текущих и потенциальных клиентов является неотъемлемой частью управления взаимоотношениями с клиентами.

Возможность точного, а также эффективного определения потребности клиентов и, в результате, выдачи им рекламы товаров, которые они сочтут желательными, открывает огромные возможности для роста бизнеса.

Применение нейронных сетей в маркетинге позволит предложить клиентам наиболее подходящие товары, рекламные продукты и услуги, что усилит эффективность методов стимулирования продаж. Это будет способствовать стабильной работе предприятия на рынке в условиях жесткой конкуренции, неопределенности и значительного влияния внешних факторов на его операции.

Ключевой целью данного проекта является изучение алгоритмов, которые можно применить для кластеризации пользователей социальных сетей.

Для достижения поставленной цели необходимо решить следующие задачи:
\begin{itemize}[leftmargin=1.6\parindent]
	\item[1)] изучить существующие алгоритмы кластеризации;
	\item[2)] выбрать алгоритмы кластеризации для кластеризации пользователей социальных сетей;
	\item[3)] изучить существующие нейронные сети и их применение в кластеризации;
    \item[4)] выбрать нейронные сети для кластеризации пользователей социальных сетей.
\end{itemize}

\pagebreak