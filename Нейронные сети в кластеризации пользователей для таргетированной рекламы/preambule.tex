%----------------------- Преамбула -----------------------
\documentclass[ut8x, 14pt, oneside, a4paper]{extarticle}

\usepackage{extsizes} % Для добавления в параметры класса документа 14pt
\usepackage{times}
% Для работы с несколькими языками и шрифтом Times New Roman по-умолчанию
\usepackage[english,russian]{babel}
\usepackage{fontspec}
%\setmainfont{timesnewromanpsmt.ttf}
\setmainfont{Times New Roman}

% ГОСТовские настройки для полей и абзацев
\usepackage[left=30mm,right=10mm,top=20mm,bottom=20mm]{geometry}
\usepackage{misccorr}
\usepackage{indentfirst}
\usepackage{enumitem}
\setlength{\parindent}{1.25cm}
%\setlength{\parskip}{1em} % поменять
%\linespread{1.3}
\renewcommand{\baselinestretch}{1.5}
\setlist{nolistsep} % Отсутствие отступов между элементами \enumerate и \itemize

% Дополнительное окружения для подписей
\usepackage{array}
\newenvironment{signstabular}[1][1]{
	\renewcommand*{\arraystretch}{#1}
	\tabular
}{
	\endtabular
}

% Переопределение стандартных \section, \subsection, \subsubsection по ГОСТу;
% Переопределение их отступов до и после для 1.5 интервала во всем документе
\usepackage{titlesec}

\titleformat{\section}[block]
{\bfseries\normalsize\filcenter}{\thesection}{1em}{}

\titleformat{\subsection}[hang]
{\bfseries\normalsize}{\thesubsection}{1em}{}
\titlespacing\subsection{\parindent}{\parskip}{\parskip}

\titleformat{\subsubsection}[hang]
{\bfseries\normalsize}{\thesubsubsection}{1em}{}
\titlespacing\subsubsection{\parindent}{\parskip}{\parskip}

% Работа с изображениями и таблицами; переопределение названий по ГОСТу
\usepackage{caption}
\captionsetup[figure]{name={Рисунок},labelsep=endash}
\captionsetup[table]{singlelinecheck=false, labelsep=endash}

\usepackage{graphicx}
\usepackage{slashbox} % Диагональное разделение первой ячейки в таблицах

% Цвета для гиперссылок и листингов
\usepackage{color}

% Гиперссылки \toc с кликабельностью
\usepackage{hyperref}

\hypersetup{
	linktoc=all,
	linkcolor=black,
	colorlinks=true,
}

% Листинги
%\setsansfont{Arial}
%\setmonofont{Courier New}

\usepackage{color} % Цвета для гиперссылок и листингов
%\definecolor{comment}{rgb}{0,0.5,0}
%\definecolor{plain}{rgb}{0.2,0.2,0.2}
%\definecolor{string}{rgb}{0.91,0.45,0.32}
%\hypersetup{citecolor=blue}
\hypersetup{citecolor=black}

\usepackage{listings}
\usepackage{caption}

\usepackage{courier}
\usepackage{wrapfig}

\usepackage{xcolor}
\captionsetup[lstlisting]{singlelinecheck=off, justification=raggedright}


\definecolor{codegreen}{rgb}{0,0.6,0}
\definecolor{codegray}{rgb}{0.5,0.5,0.5}
\definecolor{codepurple}{rgb}{0.58,0,0.82}
\definecolor{backcolour}{rgb}{0.95,0.95,0.92}

% Значения по умолчанию
\lstset{
	% подсветка синтаксиса
	backgroundcolor=\color{backcolour},   
	commentstyle=\color{codegreen},
	keywordstyle=\color{magenta},
	numberstyle=\tiny\color{codegray},
	stringstyle=\color{codepurple},
	basicstyle= \footnotesize,
	breakatwhitespace=true,% разрыв строк только на whitespacce
	breaklines=true,       % переносить длинные строки
	%   captionpos=b,          % подписи снизу -- вроде не надо
	inputencoding=koi8-r,
	numbers=left,          % нумерация слева
	numberstyle=\footnotesize,
	showspaces=false,      % показывать пробелы подчеркиваниями -- идиотизм 70-х годов
	showstringspaces=false,
	showtabs=false,        % и табы тоже
	stepnumber=1,
	tabsize=4,              % кому нужны табы по 8 символов?
	frame=single,
	escapeinside={(*}{*)}, %выделение
	xleftmargin=20.4pt,
	xrightmargin=3.4pt,
	numbersep=5pt,
	literate={а}{{\selectfont\char224}}1
	{б}{{\selectfont\char225}}1
	{в}{{\selectfont\char226}}1
	{г}{{\selectfont\char227}}1
	{д}{{\selectfont\char228}}1
	{е}{{\selectfont\char229}}1
	{ё}{{\"e}}1
	{ж}{{\selectfont\char230}}1
	{з}{{\selectfont\char231}}1
	{и}{{\selectfont\char232}}1
	{й}{{\selectfont\char233}}1
	{к}{{\selectfont\char234}}1
	{л}{{\selectfont\char235}}1
	{м}{{\selectfont\char236}}1
	{н}{{\selectfont\char237}}1
	{о}{{\selectfont\char238}}1
	{п}{{\selectfont\char239}}1
	{р}{{\selectfont\char240}}1
	{с}{{\selectfont\char241}}1
	{т}{{\selectfont\char242}}1
	{у}{{\selectfont\char243}}1
	{ф}{{\selectfont\char244}}1
	{х}{{\selectfont\char245}}1
	{ц}{{\selectfont\char246}}1
	{ч}{{\selectfont\char247}}1
	{ш}{{\selectfont\char248}}1
	{щ}{{\selectfont\char249}}1
	{ъ}{{\selectfont\char250}}1
	{ы}{{\selectfont\char251}}1
	{ь}{{\selectfont\char252}}1
	{э}{{\selectfont\char253}}1
	{ю}{{\selectfont\char254}}1
	{я}{{\selectfont\char255}}1
	{А}{{\selectfont\char192}}1
	{Б}{{\selectfont\char193}}1
	{В}{{\selectfont\char194}}1
	{Г}{{\selectfont\char195}}1
	{Д}{{\selectfont\char196}}1
	{Е}{{\selectfont\char197}}1
	{Ё}{{\"E}}1
	{Ж}{{\selectfont\char198}}1
	{З}{{\selectfont\char199}}1
	{И}{{\selectfont\char200}}1
	{Й}{{\selectfont\char201}}1
	{К}{{\selectfont\char202}}1
	{Л}{{\selectfont\char203}}1
	{М}{{\selectfont\char204}}1
	{Н}{{\selectfont\char205}}1
	{О}{{\selectfont\char206}}1
	{П}{{\selectfont\char207}}1
	{Р}{{\selectfont\char208}}1
	{С}{{\selectfont\char209}}1
	{Т}{{\selectfont\char210}}1
	{У}{{\selectfont\char211}}1
	{Ф}{{\selectfont\char212}}1
	{Х}{{\selectfont\char213}}1
	{Ц}{{\selectfont\char214}}1
	{Ч}{{\selectfont\char215}}1
	{Ш}{{\selectfont\char216}}1
	{Щ}{{\selectfont\char217}}1
	{Ъ}{{\selectfont\char218}}1
	{Ы}{{\selectfont\char219}}1
	{Ь}{{\selectfont\char220}}1
	{Э}{{\selectfont\char221}}1
	{Ю}{{\selectfont\char222}}1
	{Я}{{\selectfont\char223}}1
}

\lstloadlanguages{
	C++
}

\usepackage{listings}

% Стиль для псевдокода: строчки обычно короткие, поэтому размер шрифта побольше
\lstdefinestyle{pseudocode}{
	basicstyle=\small,
	keywordstyle=\color{black}\bfseries\underbar,
	language=Pseudocode,
	numberstyle=\footnotesize,
	commentstyle=\footnotesize\it
}

% Стиль для обычного кода: маленький шрифт
\lstdefinestyle{realcode}{
	basicstyle=\scriptsize,
	numberstyle=\footnotesize
}

% Стиль для коротких кусков обычного кода: средний шрифт
\lstdefinestyle{simplecode}{
	basicstyle=\footnotesize,
	numberstyle=\footnotesize
}

% Стиль для BNF
\lstdefinestyle{grammar}{
	basicstyle=\footnotesize,
	numberstyle=\footnotesize,
	stringstyle=\bfseries\ttfamily,
	language=BNF
}

% Определим свой язык для написания псевдокодов на основе Python
\lstdefinelanguage[]{Pseudocode}[]{Python}{
	morekeywords={each,empty,wait,do},% ключевые слова добавлять сюда
	morecomment=[s]{\{}{\}},% комменты {а-ля Pascal} смотрятся нагляднее
	literate=% а сюда добавлять операторы, которые хотите отображать как мат. символы
	{->}{\ensuremath{$\rightarrow$}~}2%
	{<-}{\ensuremath{$\leftarrow$}~}2%
	{:=}{\ensuremath{$\leftarrow$}~}2%
	{<--}{\ensuremath{$\Longleftarrow$}~}2%
}[keywords,comments]

% Свой язык для задания грамматик в BNF
\lstdefinelanguage[]{BNF}[]{}{
	morekeywords={},
	morecomment=[s]{@}{@},
	morestring=[b]",%
	literate=%
	{->}{\ensuremath{$\rightarrow$}~}2%
	{*}{\ensuremath{$^*$}~}2%
	{+}{\ensuremath{$^+$}~}2%
	{|}{\ensuremath{$|$}~}2%
}[keywords,comments,strings]

\colorlet{punct}{red!60!black}
\definecolor{background}{HTML}{EEEEEE}
\definecolor{delim}{RGB}{20,105,176}
\colorlet{numb}{magenta!60!black}
% Свой язык для json
\lstdefinelanguage[]{jsonSchema}[]{}{
	morekeywords={\$schema,type,properties,required,items,default,description,enum,\$ref,anyOf,additionalProperties,const},
	literate=
	*{0}{{{\color{numb}0}}}{1}
	{1}{{{\color{numb}1}}}{1}
	{2}{{{\color{numb}2}}}{1}
	{3}{{{\color{numb}3}}}{1}
	{4}{{{\color{numb}4}}}{1}
	{5}{{{\color{numb}5}}}{1}
	{6}{{{\color{numb}6}}}{1}
	{7}{{{\color{numb}7}}}{1}
	{8}{{{\color{numb}8}}}{1}
	{9}{{{\color{numb}9}}}{1}
	{:}{{{\color{punct}{:}}}}{1}
	{,}{{{\color{punct}{,}}}}{1}
	{\{}{{{\color{delim}{\{}}}}{1}
	{\}}{{{\color{delim}{\}}}}}{1}
	{[}{{{\color{delim}{[}}}}{1}
	{]}{{{\color{delim}{]}}}}{1},
}[keywords,comments,strings]

% Подписи к листингам на русском языке.
\renewcommand\lstlistingname{\cyr\CYRL\cyri\cyrs\cyrt\cyri\cyrn\cyrg}
\renewcommand\lstlistlistingname{\cyr\CYRL\cyri\cyrs\cyrt\cyri\cyrn\cyrg\cyri}

\DeclareCaptionLabelSeparator{line}{\ --\ }
\DeclareCaptionFont{white}{\color{white}}
\DeclareCaptionFont{black}{\color{black}}
\DeclareCaptionFormat{listing}{\colorbox[cmyk]{0,0,0,0.01}{\parbox{\textwidth}{\hspace{15pt}#1#2#3}}}
\captionsetup[lstlisting]{
	format=listing,
	labelfont=black,
	textfont=black,
	singlelinecheck=false,
	margin=0pt,
	% font={bf,footnotesize},
	labelsep=line
}

\usepackage{ulem} % Нормальное нижнее подчеркивание
\usepackage{hhline} % Двойная горизонтальная линия в таблицах
\usepackage[figure,table]{totalcount} % Подсчет изображений, таблиц
\usepackage{rotating} % Поворот изображения вместе с названием
\usepackage{lastpage} % Для подсчета числа страниц

\makeatletter
\renewcommand\@biblabel[1]{#1.}
\makeatother

\addto\captionsrussian{\def\refname{СПИСОК ИСПОЛЬЗОВАННЫХ ИСТОЧНИКОВ}}

\usepackage{color}
\usepackage[cache=false, newfloat]{minted}
\newenvironment{code}{\captionsetup{type=listing}}{}
\SetupFloatingEnvironment{listing}{name=Листинг}

\usepackage{amsmath}
\usepackage{slashbox}
