\section*{Литературный обзор}

Кластеризация является важной задачей в области анализа данных, и в последнее время нейронные сети привлекают все большее внимание исследователей в контексте решения этой задачи. В данном литературном обзоре мы рассмотрим основные работы, которые заложили основу исследования использования нейронных сетей для кластеризации, и проанализируем их вклад в развитие данной темы.

\textbf{Рульков В.С. НЕЙРОННЫЕ СЕТИ В СФЕРЕ ИНТЕРНЕТ -- МАРКЕТИНГА}

В данной статье \cite{rulkov2018neural} рассматривается понятие нейронной сети и ее применение в маркетинге. Авторы отмечают, что нейроны выполняют функцию получения информации, проведения простых расчетов и передачи результатов. Нейроны могут быть входными, невидимыми или выходными, а синапсы соединяют нейроны между собой.

Также в данной статье описываются примеры использования нейронных сетей интернет-магазинами для предложения клиентам соответствующих товаров или услуг, а также возможность распознавания голоса и фотографий в устройствах и социальных сетях.

\textbf{Sinaga,KristinaP. Unsupervised K-means clustering algorithm}

В данной статье \cite{sinaga2020unsupervised} описывается алгоритм кластеризации k-means, который является одним из наиболее известных и используемых методов кластеризации. Основные преимущества U-kmeans заключаются в возможности работы без явных инициализаций и выбора параметров. Он также обладает устойчивостью к различным объемам и формам кластеров, одновременно автоматически определяя количество кластеров. Предложенный алгоритм был протестирован на нескольких синтетических и реальных наборах данных, а также сравнен с другими существующими алгоритмами, такими как R-EM, C-FS, k-means с известным числом кластеров, k-means+gap и X-means.

\textbf{Scalable k-means++}

В данной статье \cite{bahmani2012scalable} исследуется эффективная параллельная версия алгоритма k-means++. Авторы отмечают, что k-means является одним из наиболее популярных алгоритмов обработки данных, и правильная инициализация играет важную роль в получении хорошего решения. В статье предложен алгоритм k-means||, который позволяет существенно сократить количество проходов, необходимых для получения хорошей инициализации в параллельном режиме. 